\chapter{Summary $\&$ Conclusions}
\label{chap:Conclusion}
\gls{deap3} is currently undergoing the final stages of argon filling and cooling. The detector is projected to take first data as early as Summer 2016. A functioning trigger capable of reducing the high natural $\beta$ background in atmospheric argon has been developed, debugged, and shown to function as designed. The efficiency and operation of the trigger has been studied and found to be well characterized by a Gaussian error function allowing accurate uncertainty calculations once data cuts are made. 

Slight divergence from a Gaussian model of the trigger threshold have been observed at low energy. It is expected however, that the large \gls{psd} separation between the background events and the \gls{wimp} region of interest will mean this slight shift to higher efficiency will not affect the mitigation of unwanted backgrounds as a large safety factor can be introduced. The final threshold definitions will take into account the relation between detector downtime due to the system latency while recording events and the probability of miscategorizing a non-$\beta$ event following the fit efficiency \gls{erf}. A good separation in the \gls{fprompt} and charge of these events will allow this uncertainty to be minimized, maximizing the detectors sensitivity.

\section{Future Work}
A new clock distribution system which will improve timing issues in the data acquisition system has been developed. The \gls{dtm} with the new clock distribution has yet to be thoroughly tested and will need to be studied in the same manner as this study follows for the current trigger. Although no functional change is expected in the trigger itself, this has not yet been shown. Future work will go into ensuring the correct operation and stability of the system. 

Additional development of the trigger may be necessary, such as protection against buffer overflow in the case of high rates. Trigger development and improvements will continue as needed but the currently installed trigger is functioning. It is also necessary to calibrate the energy and \gls{fprompt} levels for proper event selection which will be done once the rate of $\beta$ events in the detector become appreciable as argon is added. A similar study to the energy thresholds must be done for the \gls{fprompt} thresholds. The efficiency \gls{erf} fits will be used as possible sources of detection uncertainty once these thresholds are defined.