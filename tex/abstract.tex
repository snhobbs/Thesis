% (This is included by thesis.tex; you do not latex it by itself.)

\begin{abstract}

\gls{deap3} is a particle detector looking for weakly interacting massive particles (\gls{wimp}s) as a source of dark matter using a 3600 kg atmospheric argon target. Incident particles colliding with the argon should produce excited dimers which decay releasing scintillation light. Different particle characteristics determine the proportion of excited states of the dimers produced, and events can be characterized by the timing of the light released. Atmospheric argon produces $\sim1$ Hz/kg of $^{39}$Ar $\beta-$decay. To reduce the background rate from $\beta-$decay a trigger is introduced in order to remove a proportion of these backgrounds through pulse shape discrimination techniques. It is necessary that possible \gls{wimp} events and rare background events are not miscategorized as $\beta-$decays, so careful calibration is essential so as not to reduce the sensitivity of the experiment. This project describes the design and development of this trigger algorithm followed by thorough testing and characterization of the trigger's behaviour. The trigger is now operational and has been installed in the detector. Studies of the trigger's behaviour indicate its functionality. A study of the efficiency in the turn-on zones is also presented. 


\end{abstract}
