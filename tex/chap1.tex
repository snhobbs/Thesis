\chapter{Introduction}
\gls{deap3} is a direct dark matter detector that is currently in the last stages of development. \gls{deap3} uses liquid argon as a scintillation target looking for incoming weakly interacting dark matter particles. Projected to be the first detector sensitive enough to probe the constrained minimum supersymmetric model's (cMSSM) neutrilino mass and cross-section predictions, \gls{deap} will either give evidence in support of this theory or set a new limit on the mass and cross-section for particle dark matter.

The triggering system of \gls{deap3} is a complex system that controls the data recording electronics. Due to high rates of $\beta$-decay in atmospheric argon, the trigger must preform dynamic event rejection in such a way as to maximize the sensitivity of the detector while reducing this background. A robust, reliable and well understood trigger is a crucial component of high sensitivity particle detectors and \gls{deap3} is no exception. This project is the development and study of the triggering system. The design and motivation is covered as well as the development process and testing. Testing of the triggers efficiency and behaviour in rejecting and accepting events is covered.

This project is comprised of two primary parts: firmware development and testing and trigger analysis. The trigger analysis looks at data taken using light injection to characterize the behaviour of the triggering system when installed on the detector.

 

